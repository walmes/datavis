%-----------------------------------------------------------------------

\usepackage[math]{iwona}

\usepackage[T1]{fontenc}
\usepackage{csquotes}

% Font for code. ----------------------------
\usepackage[scaled=0.8]{beramono}

%-----------------------------------------------------------------------

% \usepackage{lmodern}
\usepackage{amssymb, amsmath}
\setcounter{MaxMatrixCols}{20} % bmatrix, max number of columns.

% \usepackage{ifxetex, ifluatex}
\usepackage{fixltx2e} % provides \textsubscript
\usepackage[utf8]{inputenc}
\usepackage[shorthands=off,main=brazil]{babel}
\usepackage{graphicx}

\usepackage{pdfpages}
\usepackage{xcolor}
\usepackage{setspace}
\usepackage{xspace}
\usepackage{tabularx}

\setlength{\parindent}{0pt}
\setlength{\parskip}{6pt plus 2pt minus 1pt}
\setlength{\emergencystretch}{3em}  % prevent overfull lines
\providecommand{\tightlist}{%
  \setlength{\itemsep}{0pt}\setlength{\parskip}{0pt}}
\setcounter{secnumdepth}{0}

% %-----------------------------------------------------------------------
% % Bibliography.
%
% \usepackage[style=authoryear]{biblatex}
% % Use:
% %   \cite{<ref>}
% %   \parencite{<ref>}
% %   \fullcite{<ref>}
% %   \footfullcite{<ref>}
% % ATTENTION: Compilation = pdflatex > biber > pdflatex > pdflatex.
% % Call refs.bib at preamble with: \addbibresource{refs.bib}
% %
%-----------------------------------------------------------------------

\usepackage[hang]{caption}
\captionsetup{font=footnotesize,
  labelfont=footnotesize,
  labelsep=period}
\usepackage{subfig}

\makeatletter
\let\@@magyar@captionfix\relax
\makeatother

\providecommand{\tightlist}{%
  \setlength{\itemsep}{0pt}\setlength{\parskip}{0pt}}

%-----------------------------------------------------------------------

\usepackage{tikz}

\usebackgroundtemplate{
  \tikz[overlay, remember picture]
  \node[% opacity=0.3,
        at=(current page.south east),
        anchor=south east,
        inner sep=0pt] {
          \includegraphics[height=\paperheight, width=\paperwidth]{config/ufpr-imprensa-blur-4x3.jpg}};
}

%-----------------------------------------------------------------------
% Definições de esquema de cores.

% Ubuntu.
% http://www.color-hex.com/color-palette/2018
\definecolor{mycolor1}{HTML}{5E2750} % Título.
\definecolor{mycolor2}{HTML}{333333} % Texto.
\definecolor{mycolor3}{HTML}{DD4814} % Estrutura.
\definecolor{mycolor4}{HTML}{DD4814} % Links.
\definecolor{mycolor5}{HTML}{AEA79F} % Preenchimentos.

\hypersetup{
  colorlinks=true,
  linkcolor=mycolor4,
  urlcolor=mycolor1,
  citecolor=mycolor1
}

%-----------------------------------------------------------------------
% ATTENTION: http://www.cpt.univ-mrs.fr/~masson/latex/Beamer-appearance-cheat-sheet.pdf

\setbeamertemplate{caption}[numbered]
\setbeamertemplate{caption label separator}{: }
\setbeamertemplate{caption}[numbered]{}
\setbeamertemplate{section in toc}[sections numbered]
\setbeamertemplate{subsection in toc}[subsections numbered]
\setbeamertemplate{sections/subsections in toc}[ball]{}
\setbeamertemplate{navigation symbols}{}
\setbeamertemplate{frametitle continuation}{\gdef\beamer@frametitle{}\vspace*{1ex}}

\setbeamertemplate{enumerate items}[default]
\setbeamertemplate{itemize items}{\scriptsize\raise1.25pt\hbox{\donotcoloroutermaths$\blacktriangleright$}}

% Rodapé.
\setbeamercolor{title in head/foot}{parent=subsection in head/foot}
\setbeamercolor{author in head/foot}{bg=mycolor4, fg=white}
\setbeamercolor{date in head/foot}{parent=subsection in head/foot, fg=mycolor3}

% Cabeçalho.
\setbeamercolor{section in head/foot}{bg=mycolor2, fg=mycolor4}
\setbeamercolor{subsection in head/foot}{bg=mycolor2, fg=white}

\setbeamercolor{title}{fg=mycolor1}       % Título dos slides.
% \setbeamercolor{titlelike}{fg=title}
\setbeamercolor{subtitle}{fg=mycolor3}    % Subtítulo.
% \setbeamercolor{author}{}
\setbeamercolor{institute}{fg=mycolor2}   % Instituição.
\setbeamercolor{frametitle}{fg=mycolor1}  % De quadro.
\setbeamercolor{structure}{fg=mycolor3}   % Listas e rodapé.
\setbeamercolor{item projected}{bg=mycolor2}

% Blocos.
% \setbeamertemplate{blocks}[rounded]
% \setbeamercolor{block title}{bg=mycolor5!75!white, fg=mycolor3}
% \setbeamercolor{block body}{bg=mycolor5!25!white, fg=mycolor2}

\setbeamercolor{normal text}{fg=mycolor2} % Texto.
\setbeamercolor{caption name}{fg=normal text.fg}

% To remove empty brackets of \institution.
\makeatletter
\setbeamertemplate{footline}{
  \leavevmode%
  \hbox{%
    \begin{beamercolorbox}[
      wd=0.3\paperwidth, ht=2.25ex, dp=1ex, right]{author in head/foot}%
      \usebeamerfont{author in head/foot}\insertshortauthor{}\hspace*{1ex}
    \end{beamercolorbox}%
    \begin{beamercolorbox}[
      wd=0.6\paperwidth, ht=2.25ex, dp=1ex, left]{title in head/foot}%
      \usebeamerfont{title in head/foot}\hspace*{1ex}\bfseries\insertshorttitle{}
      % \usebeamerfont{title in head/foot}\hspace*{1ex}\insertframetitle{}
    \end{beamercolorbox}%
    \begin{beamercolorbox}[
      wd=0.1\paperwidth, ht=2.25ex, dp=1ex, right]{date in head/foot}%
      \bfseries\insertframenumber{}\hspace*{2ex}
    \end{beamercolorbox}
  }%
  \vskip0pt%
}
\makeatother

% %-----------------------------------------------------------------------
%
% \newcommand{\mytwocolumns}[4]{
%   % #1: Line width fraction for the left column , e.g. 0.5.
%   % #2: Line width fraction for the right column.
%   % #3: Content for the left column.
%   % #4: Content for the right column.
%   \begin{columns}[t]
%     \begin{column}{#1\linewidth} %----------- left.
%       #3
%     \end{column} %--------------------------- left.
%     \begin{column}{#2\linewidth} %----------- right.
%       #4
%     \end{column} %--------------------------- right.
%   \end{columns}
% }
%
% \newcommand{\myquote}[3]{
%   \begin{center}
%     \begin{minipage}[c]{0.19\linewidth}
%       \begin{center}
%         \includegraphics[height=2.7cm]{#1}
%       \end{center}
%     \end{minipage}
%     \begin{minipage}[c]{0.7\linewidth}
%       \begin{flushright}
%         \textit{#2}
%         \vspace{1ex}
%
%         -- #3
%       \end{flushright}
%     \end{minipage}
%   \end{center}
% }
%
% \newcommand{\hi}[1]{%
%   \textcolor{mycolor4}{#1}\xspace
% }
%
% \newcommand{\myurl}[1]{%
%   \tiny{\url{#1}}\xspace
% }
%
% \newcommand{\btn}[1]{%
%   \beamergotobutton{#1}
% }
%
% %-----------------------------------------------------------------------

\author[Walmes Zeviani $\cdot$ UFPR]{%
      \href{http://leg.ufpr.br/~walmes}{Prof.~Walmes Zeviani} \\
      \href{mailto:walmes@ufpr.br}{\small\tt walmes@ufpr.br}
}

\institute[DEST/UFPR]{
  {Laboratório de Estatística e Geoinformação}\\
  {Departamento de Estatística}\\
  {Universidade Federal do Paraná}}

%-----------------------------------------------------------------------
